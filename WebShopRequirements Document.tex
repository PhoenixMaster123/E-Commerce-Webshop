\documentclass[a4paper,12pt]{article}
\usepackage[utf8]{inputenc}
\usepackage{enumitem}
\usepackage{hyperref}

\title{E-Commerce Webshop Prototype \\
	Requirements Document}
\author{
	Group Members: \\
	Enrico Ebert, enrico.ebert@study.thws.de, 5123098 \\
	Kristian Popov, 5123029, kristian.popov@study.thws.de \\
	Glison Doci, glison.doci@study.thws.de, 5123136 \\
	Orik Mazreku, orik.mazreku@study.thws.de, 5123144
}
\date{\today}

\begin{document}
	
	\maketitle
	
\section{Project Overview}

Our e-commerce webshop has a focus on a simple and modern shopping experience. There are the essential shopping features implemented, such as searching for products, filtering by category and adding them to the cart, and we want to include some extras that improve the user experience, like the option to switch between light and dark mode.

The shop should work smoothly on both mobile devices and desktop computers, so users can shop on both devices. Also, we are planning to add user logins so people can sign up and log in.

Our main features will be things like category filtering and a search function, so users can quickly search and find what they’re looking for. The user is able to add products to the cart, update the quantities, and remove items. There will also be a checkout page. For better user experience, we will implement a switch where the user can choose between light and dark mode.

\textbf{Target Users: }  
Our target group are normal online shoppers. This includes people browsing on their laptops or using their SmartPhones.

\textbf{Application Goals:}
\begin{itemize}
	\item Build a clean and user-friendly webshop with modern design.
	\item Make product search and category filtering fast and intuitive.
	\item Allow users to create accounts, log in securely, and manage their shopping activities.
	\item Enable full shopping cart functionality: add, update, and delete items.
	\item Support both light mode and dark mode for better user comfort.
	\item Ensure a responsive design for both mobile and desktop views.
	\item Keep the scope manageable but realistic for a fully working prototype.
\end{itemize}
	
	\section{Key Features Gilson}
	\begin{itemize}
		\item \textbf{Notizen} (Must-have): E-Commerce
		Suche, Warenkorb (delete, update, add), User Login, User Sign up, Kategorie filtering, aus light mode zu dark mode. Mobile-Desktop Ansicht
		\item \textbf{Product Search and Filters} (Must-have): Allow users to search and filter products by category, price, etc.
		\item \textbf{Shopping Cart} (Must-have): Add, update, and remove products; view total cost.
		\item \textbf{Checkout Process} (Must-have): Collect shipping and payment information; confirm order.
		\item \textbf{User Authentication} (Nice-to-have): Sign up, login, and order history for registered users.
		\item \textbf{Admin Dashboard} (Nice-to-have): Manage products, view orders, update statuses.
		\item \textbf{Order Confirmation Email} (Nice-to-have): Send a confirmation email upon successful order.
	\end{itemize}
	

	\section{User Roles and Interactions}
	This section explains the different types of users who interact with our e-commerce webshop, their permissions, and the actions they can typically perform.
	\subsection*{Administrator/Admin}
	The Administrator oversees the overall management of the webshop. This role has the highest level of access and can modify nearly every aspect of the site. \\ \\
	\textbf{Typical Actions and Expected Outcomes:}
	\begin{itemize}
		\item The admin can add, update, or remove products from the catalog, and these changes will be reflected immediately for customers browsing the site.
		\item The admin can view or remove user accounts and monitor and update the status of customer orders (e.g., pending, shipped, delivered, or canceled). Additionally, the admin has access to a detailed purchase history for each customer, including their name, purchased items, purchase date, order status, and total amount spent. For example, the admin can see that \textit{User123} purchased  \textit{"Wireless Headphones"} on \textit{March 15, 2025}, for a total of \textit{129.99€}.
	\end{itemize}

	\subsection*{Customer (Registered User)}
Customers are users who have registered an account on the webshop. They can explore products, place orders, and manage their personal details and order history. \\ \\
	\textbf{Typical Actions and Expected Outcomes:}
	\begin{itemize}
  		\item Customers can browse all available products, read detailed descriptions, check prices, and view customer reviews.
    		\item They can also use the search bar or apply filters such as category, price, brand, or rating to find specific items.
    		\item Add products to the shopping cart for quick purchase.
    		\item Proceed to checkout, choose payment and shipping options, and place an order.
    		\item Update personal information, including name, email, password, and delivery addresses.
	\end{itemize}

	\subsection*{Guest User (Unregistered User)}
	A guest user can browse the webshop without creating an account, though their access is limited. \\ \\
	\textbf{Typical Actions and Expected Outcomes:}
	\begin{itemize}
		\item Can browse the products without logging in.
		\item Can register or Sign In.
		\item Can add products to the cart, but the cart will remain active only while they are on the website. Once they leave, everything will be removed. The users must have an account to purchase the products otherwise it will be not possible.
	\end{itemize}
	\section{User Stories/Use Cases´}
	\subsection*{Use Case 1: Register a New User}
	\textbf{Actor(s):} Visitor \\ \\
	\textbf{Description:} A visitor creates a new account to access personalized features \\ \\
	\textbf{Preconditions:} The user is not logged in. \\ \\
	\textbf{Main Flow:}
	\begin{enumerate}
    		\item The user clicks on the ``Sign Up'' button.
    		\item The system presents a registration form.
    		\item The user enters their name, email, password, and other required data.
    		\item The system validates the data.
    		\item The system creates a new user account.
    		\item A confirmation message or email is sent to the user.
    \end{enumerate}
    \textbf{Postconditions:} A new user is registered. \\ \\
    \textbf{Alternative Flows:} If the email exists, the system shows an error message.


    \subsection*{Use Case 2: Log in to the Account}
	\textbf{Actor(s):} Registered User \\ \\
	\textbf{Description:} A user logs into their account to access their profile and order history. \\ \\
	\textbf{Preconditions:} User is registered. \\ \\
	\textbf{Main Flow:}
	\begin{enumerate}
  		\item The user clicks on the ``Login'' button.
  		\item The system displays the login form.
  		\item The user provides email and password.
  		\item The system authenticates the user.
  		\item On success, the user is redirected to the dashboard/homepage.
	\end{enumerate}
	\textbf{Postconditions:} User is logged in. \\ \\
	\textbf{Alternative Flows:} If login credentials are invalid, an error message is shown.
	\subsection*{Use Case 3: Add/Remove Product to/from the Cart}
	\textbf{Actor(s):} Logged-in or guest user \\ \\
	\textbf{Description:} The user adds/removes a product to/from the shopping cart. \\ \\
	\textbf{Preconditions:} The product needs to exist/needs to be added or removed. \\ \\
	\textbf{Main Flow:}
	\begin{enumerate}
  		\item The user clicks on a product.
  		\item The system shows product details.
  		\item The user selects size/quantity if applicable.
  		\item The user clicks ``Add to Cart'' or ``Remove from Cart''.
  		\item The system adds or removes the item from the cart accordingly.
	\end{enumerate}
	\textbf{Postconditions:} Product is either added or removed from the cart.
	\subsection*{Use Case 4: Checkout and Place Order}
	\textbf{Actor(s):} Logged-in user \\ \\
	\textbf{Description:} The user completes the purchase of products in the cart. \\ \\
	\textbf{Preconditions:} User has items in their cart and is logged in. \\ \\
	\textbf{Main Flow:}
	\begin{enumerate}
  		\item The user views the cart and clicks ``Checkout.''
  		\item The system prompts for a shipping address and payment method.
  		\item The user enters or confirms the information.
  		\item The system verifies payment and places the order.
  		\item An order confirmation page is shown.
	\end{enumerate}
	\textbf{Postconditions:} Order is saved, and confirmation is sent.
	\subsection*{Use Case 5: View Order History}
	\textbf{Actor(s):} Logged-in user \\ \\
	\textbf{Description:} The user views a list of their past orders. \\ \\
	\textbf{Preconditions:} The user has placed at least one order. \\ \\
	\textbf{Main Flow:}
	\begin{enumerate}
  		\item The user navigates to ``My Account'' → ``Order History.''
  		\item The system displays a list of previous orders with status and details.
  		\item The user clicks on an order to view more information.
	\end{enumerate}
	\textbf{Postconditions:} User sees order details.
	\subsection*{Use Case 6: Search for a Product}
	\textbf{Actor(s):} Any user (guest or registered) \\ \\
	\textbf{Description:} The user searches for a specific product using a search bar. \\ \\
	\textbf{Preconditions:} None \\ \\
	\textbf{Main Flow:}
	\begin{enumerate}
  		\item The user types a keyword in the search bar.
  		\item The system displays matching products.
  		\item The user clicks on one to see details.
	\end{enumerate}
	\textbf{Postconditions:} Matching products are shown. \\ \\
	\textbf{Alternative Flows:} If no matches are found, the system shows nothing.
	\subsection*{Use Case 7: Apply Discount Code}
	\textbf{Actor(s):} Logged-in \\ \\
	\textbf{Description:} The user applies a discount code during checkout to receive a \\ reduction in
	price. \\ \\
	\textbf{Preconditions:} A valid discount code is available. \\ \\
	\textbf{Main Flow:}
	\begin{enumerate}
  		\item The user proceeds to checkout.
  		\item The system displays a field to enter the discount code.
  		\item The user enters the discount code and clicks ``Apply.''
  		\item The system validates the code and applies the discount to the order total.
  		\item The user reviews the updated order price.
	\end{enumerate}
	\textbf{Postconditions:} The total price is reduced by the discount.

	
	\section{Non-Functional Requirements}

Our E-Commerce WebShop is designed to offer a stable, secure, and user-friendly experience. The following non-functional requirements describe the quality standards we aim to follow throughout the development:

\begin{itemize}
    \item \textbf{Usability:}  
    \begin{itemize}
        \item \textbf{Intuitive Navigation:}  
        We want users to find their way around easily. That’s why we’re going for clear and consistent menus (using tools like \texttt{NavLink} for faster routing).

        \item \textbf{User-Friendly Design:}  
        The design will be clean and modern, with a focus on readability. We’ll stick to a consistent color scheme, readable fonts, and buttons that look and feel familiar, so users don’t have to think twice about how to interact with the site.

        \item \textbf{Efficient Checkout Process:}  
        No one likes a complicated checkout. Progress indicators like “Step 1 of 3: Shipping” will help users know where they are in the process and what’s next.
    \end{itemize}

    \item \textbf{Responsiveness:}  
    Whether users are shopping on a desktop, tablet, or phone, the webshop should feel natural to use. Everything should adapt cleanly to different screen sizes without breaking the layout.

    \item \textbf{Performance:}  
    \begin{itemize}
        \item \textbf{Loading Speed:}  
        Nobody likes waiting, so we’re aiming for pages that load in under two seconds for most users. Images that aren’t immediately visible will be loaded lazily to avoid slowing things down.

        \item \textbf{Optimized Assets:}  
        To keep things fast, we’ll compress images (using formats like WebP where possible).

        \item \textbf{Smooth Interactions:}  
        Features like search should feel instant. We'll debounce inputs to avoid spamming the backend with requests, and we’ll cache frequently used data like product listings on the client side so everything feels snappy.
    \end{itemize}

    \item \textbf{Maintainability:}  
    We’re keeping things modular and organized. Reusable components and a clean file structure will make it easier to make changes later or onboard new team members if needed.

    \item \textbf{Error Handling:}  
        When something goes wrong, we’ll show helpful messages—like “Item out of stock—try similar products”—instead of confusing errors. If an API call fails, we’ll display fallback content and a “Retry” button where it makes sense.
    
    \item \textbf{Security:}  
    Since we’re handling user input and possibly sensitive data, we’ll validate inputs and follow common security practices to prevent things like cross-site scripting (XSS) or other web attacks.

    \item \textbf{Scalability:}  
    The webshop shouldn’t feel limited. As the number of users and products grows, the system should still perform well without requiring major rewrites.

    \item \textbf{Reliability:}  
    The app should work as expected—even under load. During sales or other busy times, it needs to stay up and running smoothly without crashing or freezing.

    \item \textbf{Internationalization}
    \begin{itemize}
        \item \textbf{Multi-Language Support:}  
        Even if we only launch in one language at first, we’ll prepare the app with translation files (like \texttt{en.json} and \texttt{de.json}) so we can easily add support for other languages later.

        \item \textbf{Localized Content:}  
        Things like currency, dates, and measurements should adapt to where the user is from, so the experience feels natural to them no matter the region.
    \end{itemize}
\end{itemize}
	
	\section{Technology Assumptions// machen wir später}
	\begin{itemize}
		\item \textbf{Frontend:} React.js (mandatory for this course).
	
		\item \textbf{API Communication:} RESTful API for communication between frontend and backend.
		\item \textbf{Database:} MongoDB or PostgreSQL for storing product and order data.
		\item \textbf{Styling:} Tailwind CSS for fast and responsive UI design.
		\item \textbf{Email Service:} (Optional) Use an external service like SendGrid for order confirmations.
	\end{itemize}
	
	\section{Project Constraints//machen wir später }
	\begin{itemize}
		\item \textbf{Time Constraint:} Completion of the project before end of semester, with focus on core features first.
		\item \textbf{Resource Constraint:} Small team size, limited development hours per week.
		\item \textbf{Technical Constraint:} Mandatory use of React.js for frontend; backend technology choice is flexible.
		\item \textbf{Open Questions:}
		\begin{itemize}
			\item Will we integrate payment gateways (like Stripe) or simulate payments?
			\item Scope of admin features: full CRUD or minimal inventory management?
			\item Do we include user profiles and order history or focus solely on guest checkout?
		\end{itemize}
	\end{itemize}
	
	\section{Acknowledgment of AI Technologies}
	This document was drafted and refined using GPT-4o based on an outline containing related information. The authors reviewed, revised, and enhanced the GPT-4o output with additional content. The document was then edited for improved readability and active tense, partially using Grammarly.
	
\end{document}
